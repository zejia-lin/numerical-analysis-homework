
\section{引言}

使用参考文献的示例\cite{vaswani2017attention}\cite{DBLP:journals/corr/abs-1802-05751}\cite{mott2019towards}\cite{silver2016mastering}.

插入代码的示例:

\begin{lstlisting}[style=sysupython]
def main():
    print("hello world")
\end{lstlisting}
代码高亮风格可在\texttt{sysulab.sty}中修改.
\begin{lstlisting}[style=sysucpp]
#include <iostream>
using namespace std;
int main() {
    printf(123);
    return 0;
}
\end{lstlisting}

伪代码示例:


\begin{algorithm}[!htbp]
    \caption{水印嵌入算法}
    \label{algo:embed}
    \begin{algorithmic}[1] %每行显示行号
        \Require $A\ (m\times n)$图像矩阵, $W\ (s\times t)$水印矩阵,$C_{mod}$求余常数
        \Ensure $\hat{A}\ (m\times n)$含水印图像矩阵
        \Function {EmbedWatermark}{$A,\ W,\ C_{mod}$}
        \State $cA, cH, cV, cD \gets \Call{computeDWT}{A} $ \Comment{获取DWT各分量}
        \State $numBlocks \gets \left\lfloor\dfrac{m}{8}\right\rfloor\cdot \left\lfloor\dfrac{n}{8}\right\rfloor$ \Comment{获取分块数量}
        \ParFor {$i = 0 \to numBlocks$} \Comment{并行计算每个$4\times 4$分块}
        \State $D \gets \Call{computeDCT}{cA_i} $ \Comment{计算DCT}
        \State $U, S, V \gets \Call{computeSVD}{D}$ \Comment{计算SVD}
        \State $b \gets W_{i \mod numBlocks}$ \Comment{对水印进行循环嵌入,获取该分块嵌入的水印1bit}
        \State $\hat{S_0} = \left(\left\lfloor\dfrac{S_0}{mod}\right\rfloor + \dfrac{1}{4} + \dfrac{b}{2} \right) * C_{mod}$ \Comment{将该bit嵌入到最大奇异值中}
        \State $\hat{D} \gets U \cdot \hat{S} \cdot V^T$ \Comment{还原DCT系数矩阵}
        \State $\hat{cA_i} \gets \Call{computeIDCT}{\hat{D}}$ \Comment{DCT逆变换}
        \EndParFor
        \State $\hat{A} \gets \Call{computeIDWT}{\hat{cA}, cH, cV, cD}$ \Comment{DWT逆变换}
        \State \Return $\hat{A}$
        \EndFunction
    \end{algorithmic}
\end{algorithm}
